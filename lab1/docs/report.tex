\documentclass[12pt]{article}

\usepackage{fullpage}
\usepackage{multicol,multirow}
\usepackage{tabularx}
\usepackage{ulem}
\usepackage[utf8]{inputenc}
\usepackage[russian]{babel}
\usepackage{pgfplots}

% Оригиналный шаблон: http://k806.ru/dalabs/da-report-template-2012.tex

\begin{document}

\section*{Лабораторная работа №\,1 по курсу дискрeтного анализа: сортировка за линейное время}

Выполнил студент группы 08-208 МАИ \textit{Попов Николай}.

\subsection*{Условие}

Кратко описывается задача: 
\begin{enumerate}
\item 

Требуется разработать программу, осуществляющую ввод пар «ключ-значение», их упорядочивание по возрастанию ключа указанным алгоритмом сортировки за линейное время и вывод отсортированной последовательности.

\item H. 3-2 

Поразрядная сортировка.

Тип ключа: числа от 0 до $2^{64}-1$  \\
Тип значения: строки переменной длины (до 2048 символов).

\end{enumerate}

\subsection*{Метод решения}

Поразрядная сортировка имеет 2 версии, в этой лабораторной я решил использовать
LSD (Least Significant Digit radix sort). В качестве устойчивой сортировки для сортировки 
элементов одного разряда была выбрана сортировка подсчетом.

\subsection*{Описание программы}

Для хранения элемента была создана структура, состоящая из двух полей: ключа и значения.
Тип данных для ключа $uint64_t$ позволяет хранить всевозможные значения ключа указанные в задании. 
Для хранения строки-значения была создана структура, хранящая длину строки, массив элементов типа $char$.


\subsection*{Дневник отладки}

При выводе результата не учитывалось время, требуемое для очистки буфера при использовании 
$std::endl$

\subsection*{Тест производительности}

\begin{figure}
\centering
\begin{tikzpicture}
\begin{axis}[
    width=10cm,
    xlabel={N input data size},
    ylabel={execution time, sec},
    title={Radix Sort},
    xmin=0, xmax=1e6,
    ymin=0, ymax=10,
    xtick={0, 2e5, 4e5, 6e5, 8e5, 1e6},
    ytick={0, 1, 2, 3, 4, 5, 6, 7, 8, 9},
    legend pos=north west,
    grid=both,
]
\addplot[color=blue,mark=*] coordinates {
    (0, 0)
    (2e5, 2.2)
    (4e5, 3.7)
    (6e5, 4.8)
    (8e5, 6.3)
    (1e6, 7.8)
};
\end{axis}
\end{tikzpicture}
\caption{Execution Time vs. Input Data Size}
\end{figure}


Оценка сложности: O(n) где , n - количество элементов в массиве.
Алгоритм линеен относительно количества входных данных.
Как видно из графика, рост времени работы при увеличении объема входных данных
в среднем увеличивается линейно.

\subsection*{Выводы}

Выполняя лабораторную работу по курсу «Дискретный анализ», я научился реализовывать 
поразрядную сортировку и сортировку подсчётом,
вспомнил работу с памятью. Это поможет мне в ситуации, когда нужно будет написать
быструю сортировку, которая будет работать за линейное время. 

\end{document}

